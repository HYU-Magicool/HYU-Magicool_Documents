\documentclass[conference]{IEEEtran}
\usepackage{graphicx}
\IEEEoverridecommandlockouts
% The preceding line is only needed to identify funding in the first footnote. If that is unneeded, please comment it out.
\usepackage{cite}
\usepackage{amsmath,amssymb,amsfonts}
\usepackage{algorithmic}
\usepackage{graphicx}
\usepackage{textcomp}
\usepackage{xcolor}
\def\BibTeX{{\rm B\kern-.05em{\sc i\kern-.025em b}\kern-.08em
    T\kern-.1667em\lower.7ex\hbox{E}\kern-.125emX}}

\begin{document}

\title{Freshkeeper\\
{\footnotesize \large Mini portable refrigerator application for the freshness of early morning delivery with LG Refrigerators}}

\author{\IEEEauthorblockN{1\textsuperscript{st} Lee JunYoung}
\IEEEauthorblockA{\textit{dept. of Information Systems} \\
\textit{Hanyang University}\\
Seoul, Republic of Korea \\
leewnsdud123@gmail.com}
\and
\IEEEauthorblockN{2\textsuperscript{nd} Park SeokWon}
\IEEEauthorblockA{\textit{dept. of Information Systems} \\
\textit{Hanyang University}\\
Seoul, Republic of Korea \\
xylitolfrog@gmail.com}
\and
\IEEEauthorblockN{3\textsuperscript{rd} Chung YouBeen}
\IEEEauthorblockA{\textit{dept. of Information Systems} \\
\textit{Hanyang University}\\
Seoul, Republic of Korea \\
youbeen0308@gmail.com}
}
\maketitle

\begin{abstract}
Freshkeeper is a mobile application service designed to address the concerns people encounter when using courier delivery: ensuring the preservation of refrigerated food that being delivered. This platform empowers individuals to make sustainable choices while enjoying the convenience of courier services and keeping their food fresh. Freshkeeper is also designed to address the environmental concerns that emerge during the courier delivery, since waste such as ice packs are generated during courier delivery to maintain the freshness of food. The Freshkeeper service offers a comprehensive solution for users by combining the following key features: Freshkeeper encourages sustainable courier delivery practices by reducing unnecessary packaging,  Users can seamlessly select eco-friendly delivery options, track their packages, and minimize the environmental impact of their shipments. With our service, people can utilize AI speakers to receive voice notifications for estimated time of arrival of goods confirmations, and generative AI technology to offer healthier alternatives to the food user order. 
Freshkeeper is a service committed to promoting sustainability, reducing waste, and enhancing the overall user experience in terms of refrigerator. By utilizing this innovative service, users can actively contribute to a greener planet while enjoying the benefits of fresher food, streamlined courier delivery services, and ease of refrigerator management .  

\end{abstract}

\begin{IEEEkeywords}
refrigerator, space optimization, mobile application
\end{IEEEkeywords}

\begin{table}[ht]
\caption{Role Assignment}
\begin{tabular}{| p{1.3cm}|p{1cm}|p{5.5cm} |}

\hline
Roles & Name & Role Description\\
\hline
Development \par Manager, \par User 
& Lee \par JunYeong 
& Development Manager is responsible for creating a clear project plan. This involves defining project goals, objectives, scope, and timeline. He ensures that the project development process is being done well, and makes sure that testing and implementation of the product is well deployed. \par
Users play an indirect role in shaping the app's requirements. Their needs and preferences provide the foundation for what features and functionalities the app should have. Developers and designers rely on user feedback and market research to determine the app's scope. Users are often the primary source of feedback during the app's development and after it's released. Developers use this feedback to refine and improve the app. Also, When users encounter issues or bugs in the app, they can report these problems to the development team. This feedback is essential for resolving issues and ensuring the app's stability.\\
\hline
\end{tabular}
\end{table}
\begin{table}
\begin{tabular}{| p{1.7cm}|p{1.5cm}|p{4.3cm} |}
\hline
Software  \par Developer,  \par Customer 
& Park \par SeokWon 
& Software	 Developers are responsible for designing the overall architecture of the mobile app. They decide on the technology stack, frameworks, and the overall structure of the app.
Writing code is a primary responsibility. Developers use programming languages to create the app's functionality. They need to ensure that the code is efficient, maintainable, and follows best practices. Also, Software Developers implement the user interface (UI) and user experience (UX) components of the app. This involves creating responsive and visually appealing layouts. In addition, Software Developers are responsible for testing the app thoroughly to identify and fix bugs and issues.
\par Customer identify what they want to implement with the app when it comes to refrigerator management, and further identify and present what functions should be added to make refrigerator management more convenient and efficient. Such preferences of customers’ become the main objective for mobile app service. \\
\hline
Product \par Designer,\par Document
& Chung \par YouBeen 
& A Product Designer is responsible for design- ing applications to provide the most efficient interface from the user’s point of view. 
She designed services to meet the needs of customers. She uses Figma, a UI design tool, to produce high-quality results.
 She works with developers to think about how to reflect users and deliver functions most efficiently to them.
Product designer must focus on the usability of a product or service and modify the design of a product to obtain better results. 
She is responsible for communicating smoothly with other team members and create collaboration.
 As our project progresses, she writes and organizes all the documentation.
(other references, research documentation required for the service)
 \\
\hline
\end{tabular}
\end{table}


\section{Introduction}
Since the first one was developed, refrigerator has been an essential device that cannot be separated from human life. Nowadays, refrigerators became so important that foods without refrigerator stored is totally rarely, but a critical weakness exists: narrow space.
We could be inexperienced to arrange contents inside the refrigerator, also classify some foods that influence others, even we may forget the existence of food or miss the position.
There are certain mobile applications helping people managing their refrigerators on the market, but their functions seem to be limited to only managing expiration date of the foods inside the refrigerator. 
‘Freshkeeper’ is an app that advise how to rearrange the substances with LG refrigerators. It might be helpful for users who admire to keep foods safe and fresh or feel hard to manage refrigerator space. Freshkeeper will identify the model of refrigerator and recognize items occupying refrigerator space to offer customers recommendations how to sort items effectively or visually by the user side.



\subsection{Significance of Refrigerators in Modern Life}
    
To appreciate the value of Freshkeeper, it's important to recognize the pivotal role refrigerators play in modern life. Refrigerators have become a staple in households around the world. They have transformed the way we store and consume food, ensuring that fresh and perishable items remain safe for consumption. Refrigerators have become an integral part of our daily lives, allowing us to keep a wide range of foods and beverages readily accessible. As a result, it's rare to find a household without a refrigerator.

However, this technological marvel comes with its own set of challenges. The most prominent among these challenges is the issue of limited space. Despite the generous storage capacity of modern refrigerators, users often find it difficult to efficiently organize their food items. This can lead to several problems:

\begin{itemize}
    \item Inefficient Arrangement\par
    Users might be inexperienced in arranging the contents of their refrigerators. They may struggle to categorize different foods or may inadvertently store items in a way that influences others negatively. The consequence is a cluttered and poorly organized fridge.    
    \item Food Wastage\par
    Inefficient organization often results in users forgetting about the existence of certain food items. These items may get pushed to the back of the refrigerator, where they are eventually forgotten and go to waste. This not only leads to financial losses but also contributes to food wastage, a pressing global issue.    
    \item Difficulty in Finding Items\par
    A chaotic refrigerator can be frustrating to deal with. When you're unable to locate specific items, it not only wastes time but can also lead to unnecessary stress and annoyance.
\end{itemize}

\subsection{Motivation}

The advent of smart home technology has brought about a paradigm shift in the way we interact with our everyday appliances. Refrigerators, once considered a humble and indispensable part of our lives, have now been reimagined as sophisticated devices in the era of LG's smarthome services. As part of this transformative journey, the development of the Freshkeeper application emerges as a remarkable innovation, one that promises to enhance the way we manage the contents of our refrigerators.

Freshkeeper is envisioned as an ingenious solution that delves into the intricate world of refrigerator management. It goes beyond the conventional purpose of a refrigerator as a mere storage space and adds a layer of intelligence to it. This application analyzes the structure of the customer's refrigerator model and calculates the optimized composition of substances within, while also providing valuable notifications about expiration dates. In doing so, it aims to revolutionize the way we consume food, fostering efficiency, health, and convenience.

\subsection{Objectives}
The objectives of Freshkeeper can be summarized in two simple yet profound words: 'optimization' and 'convenience.' Let's delve into how these objectives are achieved and the benefits they bring to users.
\begin{itemize}
    \item Optimization\par
    Freshkeeper seeks to optimize the way we use our refrigerator space. Modern refrigerators are a marvel of engineering, offering ample storage capacity, yet their potential often goes underutilized. Many users struggle with inefficiently organized contents, which not only waste space but also make it difficult to locate items when needed. Refrrange comes to the rescue by providing recommendations on how to rearrange the contents of the refrigerator to create more empty spaces, allowing users to store more items effectively. This optimization ensures that users get the most out of their refrigerator's capacity and minimizes food wastage.
    \item Convenience\par
    A cluttered refrigerator can be frustrating and chaotic to deal with. An inconvenient user interface and vague advice on food placement can exacerbate the problem. Freshkeeper takes convenience seriously by offering a user-friendly interface and clear, practical recommendations. Users are guided through the process of arranging their refrigerator, making it a hassle-free and enjoyable experience. The application aims to prevent uncomfortable instances by providing precise, easy-to-follow suggestions. The result is a refrigerator that is not only space-efficient but also a pleasure to use.
\end{itemize}

\subsection{Problem Statement}
\begin{itemize}
    \item Food Labeling and Categorization\par
    Our application should accurately categorize and label various food items to provide relevant recommendations. 
    \item Data Privacy and Security\par
    Because of the data analysis, user data collection is essential on working app normally. Security and privacy of user data should be ensured when the app deals with information such as user food inventory or consumption preference.
    \item Accuracy of Expiration Dates\par
    Expiration Dates are directly related to user health. Not to offer wrong expiration date is the most important problem Freshkeeper deals with.
    \item Notification and Reminder\par
    Timely delivery notifications and reminders is considerable in user-friendly aspect. Inform upcoming expiration dates have to be available for users.
    \item Inventory Tracking and Update\par
    Similar to expiration dates problem, tracking of refrigerator storage is in consideration how much value is on either the importance and the accuracy of inventory.
    \item User Interface and User Experience\par
    User-friendly interface and clear visualization of advice is another problem we got.
\end{itemize}

\subsection{Research on Related Software}
Samsung Family Hub\par
Samsung Family Hub is a part of Samsung's smart home ecosystem and focuses on enhancing user refrigerator experience. Family Hub offers a refrigerator management application which allows user to monitor and manage the contents of refrigerator in conjunction with the Samsung refrigerator.
\begin{itemize}
    \item Integration with Smart Refrigerators\par
    Connect with the Samsung Family Hub refrigerator to check and manage the refrigerator's contents in real-time.
    \item Ingredient Management\par
    Track the ingredients inside your refrigerator and receive notifications about expiration dates.
    \item Recipe Recommendations\par
    Based on the refrigerator's contents, it offers recipe recommendations to provide cooking ideas.
\end{itemize}
Whirlpool KitchenAid\par
Whirlpool offers a refrigerator management application that works with Whirlpool and KitchenAid appliances. The application focus on enhancing food preservation and offering convenient features. 
\begin{itemize}
    \item Refrigerator Status Monitoring\par
    Monitor the temperature and humidity of the refrigerator to maintain freshness.
    \item Expiration Date Tracking\par
    Track the expiration dates of food items in the refrigerator and provide alerts.
    \item Recipe Management\par
    Search and manage recipes based on the refrigerator's contents.
\end{itemize}
ThinQ\par
LG ThinQ is a broader application that integrates various LG smart devices, including refrigerators.
\begin{itemize}
    \item Smart Home Management\par
    LG ThinQ allows you to manage LG smart devices and refrigerators in one place.
    \item Food Management\par
    Monitor the contents of your refrigerator and receive notifications about expiration dates.
    \item Energy Savings\par
    Optimize and save energy consumption through refrigerator settings.
\end{itemize}


\section{Requirements}

\subsection{Mobile Application}
The primary requirement is the development of a mobile application specifically designed for the Android OS. The purpose of this application is to provide users with convenient access to the refrigerator management service.

\subsection{Refrigerator Registration}
The application should have the capability to retrieve information related to LG refrigerators from a central server. Alternatively, users should be able to manually register their refrigerators within the application.

\subsection{Adding Items}
The core functionality involves the addition of items to be stored within the user's refrigerator. The application should present a predefined list of commonly stored items in refrigerators, including essential details such as item name and expiration date. Additionally, users should be able to manually input items and their associated information if they are not found in the predefined list.

\subsection{Item information modification}
Users must have the option to modify the information of any specific item that they have added to their refrigerator. This editing feature ensures that the data remains accurate and up-to-date.

\subsection{Item Arrangement Helper}
The application offers a valuable feature that provides recommendations on how to arrange items within the refrigerator. These recommendations are tailored to the available space in the refrigerator and can take two forms:
\begin{itemize}
    \item Optimal Arrangement\par
    Suggests the most efficient way to store items for space optimization.
    \item Full Arrangement\par
    Recommends arrangements that maximize the capacity of the refrigerator.
\end{itemize}

\subsection{Notification}
The application includes a notification system that alerts users to the upcoming expiration dates of items stored in their refrigerators. Users can customize their notification preferences, including setting notification intervals, such as receiving alerts one week prior to an item's expiration date.

\subsection{Providing Information}
The application serves as an information resource, offering valuable insights and tips to assist users in effectively managing the contents of their refrigerators. This information can cover various aspects of refrigerator management, from food safety guidelines to best practices for organization and storage.

\section{Development Environment}
\subsection{Choice of software development platform}
Our team develops applications for both Mac and Windows operating system environments. We have selected relevant frameworks for cross-platform development. First of all, cross-platform mobile app development refers to the method of developing software on multiple different platforms or operating systems using the same code base. In the early days of cross-platform, people were reluctant to use it because the performance did not keep up with native apps, but recent technologies are approaching the level of native apps and are on their way to being recommended.  Some examples include React Native, Flutter, Xamarin, and NativeScript. These frameworks are used to develop applications that run on multiple platforms using one code base.  For this, the latest versions of Windows and MacOS are needed.  \par
Windows 11  - Windows 11 is the most recent version, offering a new look and feel to the user interface, performance improvements, and features Microsoft Teams integration and support for Android apps. \par
MacOS 12 (Monterey)  - MacOS 12 offers unified control across Apple devices with Universal Control and enables mirroring of iOS and iPadOS screens to a Mac via AirPlay to Mac. It enhances the user experience by including Focus mode, the Shortcuts app, an improved Safari browser, increased privacy, and teamwork features.

For this project, we created a team workspace to collaborate. The UI/UX design of the application will be done utilizing Figma. For the overall project schedule, execution, process, and meeting log management, we utilized Notion and shared it with team members. To communicate, we used Slack, the collaboration tool. In addition to acting as a messenger, Slack was great for uploading and sharing files.

\begin{table}[ht]
\caption{DEVELOPMENT LANGUAGE AND ENVIRONMENT}
\begin{tabular}{| p{2.9cm}|p{4.5cm} |}

\hline
Tools and Language & Reason  \\
\hline
Flutter
& First announced by Google in 2017, Flutter was originally created to run on the Fuchsia operating system, but it became a cross-platform platform when Google decided to make it compatible with other operating systems.
Because it was created by Google, it has excellent performance running on Android first, and it is truly cross-platform as it runs on iOS, Android, Windows app development for macOS, and even web browsers recently, and it is the fastest growing cross-platform.
Since Flutter can target iOS, Android, web, and desktop platforms with one codebase, we felt it would simplify development and maintenance, saving us time and money. Another key feature is that Flutter is composed of reusable UI components called widgets, which are used to build every part of the application. These widgets are rich, extensible, and help us build UIs quickly, so we felt it made sense to develop with agile methods to ensure the success of our project.\\

\hline
Node.js 
& A JavaScript-based software platform that can be used to develop server-side applications.
It runs on a variety of platforms including MS Windows, Linux, and macOS. Created with the vision of "JavaScript everywhere," the advent of Node.js made it possible to develop both web client and server applications using only the JavaScript language.
Our team felt that Node.js was an ideal choice for connecting our front-end and back-end and providing APIs, as we felt that a single language and technology stack could manage the entire stack, which would be very useful in integrating with the front-end.Node.js also has the advantage of having an active developer community and a diverse open source package ecosystem, with tons of libraries and modules readily available through its package manager, npm.
 \\
\hline
\end{tabular}
\end{table}

\begin{table}[ht]
\caption{TEAM’S WORK ENVIRONMENT}
\begin{tabular}{| p{2.9cm}|p{4.5cm} |}

\hline
Name & Environment \\
\hline
Chung YouBeen 
& Windows 11 Home with 64-bit Operating
System, and x64-based Processor, \par Jupyter Notebook \\

\hline

Park SeokWon 
& Windows 11 Pro, 64-bit Operating System, x64-based Processor \\

\hline
Lee JunYoung
& MacOS Ventura 64-bit
 \\
\hline
\end{tabular}
\end{table}

\subsection{Software in use}
\begin{itemize}
    \item Flutter \par 
    Flutter is an open-source UI framework developed by Google, designed to enable rapid development and efficient execution of mobile applications. It utilizes the Dart programming language and allows for the development of apps for both iOS and Android platforms using a single codebase. When applied to our project, Flutter provides a unified development environment that ensures consistency across iOS and Android, reducing development time and costs.
    \item Ionic \par
    Ionic is an open-source platform for building mobile and web applications, often used in conjunction with Angular for developing cross-platform apps. It is based on HTML, CSS, and JavaScript and offers a variety of plugins and tools to simplify app development. Incorporating Ionic into a collaborative project streamlines web app development and facilitates deployment to multiple platforms, enhancing developer-designer collaboration and optimizing the team's development process.
     \item Apache Cordova \par
    Apache Cordova is an open-source mobile app development framework that utilizes HTML, CSS, and JavaScript for developing mobile applications. It enables the deployment of apps to various platforms (iOS, Android, etc.) using web technologies. Cordova simplifies code sharing and collaboration among developers with different technology stacks. In a collaborative project, it enhances code sharing and streamlines the development process.
     \item AWS Lightsail \par
    AWS Lightsail is a simplified web application hosting and cloud server service provided by Amazon Web Services (AWS). It allows users to host websites and deploy applications without the need for complex configurations. When integrated into a collaborative project, Lightsail provides a hassle-free environment for deploying and scaling applications, ensuring seamless collaboration among team members.
     \item AWS \par
    Amazon Web Services (AWS) is a cloud computing and web service platform provided by Amazon, offering a wide range of cloud services to enterprises and developers. These services encompass computing, storage, databases, artificial intelligence, and various other functionalities. Incorporating AWS into collaborative projects empowers teams to efficiently manage infrastructure, share data, and strengthen collaboration among our  team.
     \item Node.js \par
    Node.js is a JavaScript runtime environment used for developing server-side applications. It is based on an asynchronous event-driven architecture and is particularly useful for creating network applications using JavaScript. In a collaborative project, Node.js facilitates effective communication with databases and ensures rapid response times, enhancing overall team productivity.
     \item PostgreSQL \par
    PostgreSQL is a high-performance, secure, and scalable open-source relational database management system (RDBMS). It supports various data types and complex queries, providing reliable data storage and management. When integrated into a collaborative project, PostgreSQL ensures data integrity and enables multiple team members to access and manipulate data.
     \item EXPO \par
    EXPO is a framework and toolset that makes it easier to develop React Native apps. Developers can quickly develop and deploy iOS and Android apps using JavaScript and React Native. EXPO expedites development and debugging, ensuring project deadlines are met.
     \item React Native \par
   React Native is an open-source framework developed by Facebook that allows for the creation of mobile apps targeting iOS and Android platforms using JavaScript and React. It enables the development of apps for multiple platforms with a single codebase. In a collaborative project, React Native simplifies code sharing and maintenance, allowing for the rapid development of cross-platform apps.
     \item Firebase \par
    Firebase is a mobile and web app development platform provided by Google, offering features such as user authentication, databases, hosting, and more. Firebase enables developers to quickly develop and operate apps. In our project, Firebase enhances data sharing and real-time collaboration among our team members.
     \item Figma \par
    Figma is a collaborative design tool for creating designs and prototypes, accessible through a web-based application. It allows multiple users to collaborate on design projects simultaneously and supports real-time co-authoring. When used in a collaborative project, Figma streamlines the design collaboration process, enabling designers and developers to share design elements and prototypes and provide feedback in real time.
    \item Notion \par
    Notion is a collaboration and productivity tool for performing various tasks, including note-taking, project management, scheduling, and more. Users can create customizable workspaces and efficiently organize multiple tasks. In a collaborative project, Notion facilitates task organization, project tracking, and team communication.
    \item Github \par
    Github is a code hosting platform based on Git, used for software development and collaboration. It supports code version control, collaboration, issue tracking, and source code hosting and is widely used in the developer community. In a collaborative project, Github simplifies code management, fosters collaboration among team members, and ensures version control and issue tracking.
     \item Overleaf \par
    Overleaf is an online LaTeX editor and collaborative environment that provides tools for writing papers and documents. It is used for creating scientific and technical documents using LaTeX and supports real-time.
    
    
\end{itemize}
\subsection{Cost Estimation}
We'll need to recreate the space for the built-in fridge near the front door (which is still temporary), so contractually secure that space when people build the building. It'll need to create a fridge slot or built-in space. There will be construction and installation costs for this, and may need professional help. There will also be parts costs to maintain the fridge, and there will be costs when the mini fridge breaks down.

From a software perspective, there may be additional costs for smart home connectivity. You need to consider the cost of setting up software and configuring apps for this connectivity, and the server infrastructure for a smart refrigerator requires web servers, databases, APIs, authentication and security, real-time communication, scalability, and back-end management and maintenance capabilities.

Since we need to get information from the server in real time, we will use AWS, a commercial cloud platform, as an example of a backend server and database that can be integrated for our prototype service. Among them, AWS Lightsail is a simplified platform for use cases such as hosting simple web applications and websites. It simplifies provisioning and management and allows users to quickly launch applications without complex setup. AWS Lightsail uses a simple monthly pricing model, and users can more easily predict their usage and charges. There is also an AWS Free Tier aimed at students and developers, which allows them to use certain AWS services for free within a limited scope.
\subsection{Task distribution}
we will provide this later at the next phase - design
but we've decided it like this (temporarily). It may change later
\begin{table}[ht]
\caption{Task distribution}
\begin{tabular}{| p{1.7cm}|p{1.5cm}|p{4.3cm} |}

\hline
Roles & Name & Description\\
\hline
Backend  \par Developer
& Lee \par JunYeong 
& A Backend Developer is responsible for the server-side or backend portion of web and mobile applications. They interact with databases, manage the application's logic, and support the actions users perform through the app. Backend Developers handle server and data processing efficiently and maintain security. \\

\hline
Frontend \par Developer
& Park \par SeokWon 
& A Frontend Developer is responsible for designing and developing the user interface (UI) of web and mobile applications. They use web technologies like HTML, CSS, and JavaScript to build the visual components of web pages and apps that users interact with, enhancing the user experience. \\
\hline
UI/UX Product \par Designer
& Chung \par YouBeen 
& A UI/UX Product Designer is responsible for user experience (UX) and user interface (UI) design. They optimize the design of applications or websites to make them intuitive for users. UX designers understand user requirements, while UI designers implement them visually.
 \\
\hline
\end{tabular}
\end{table}

\section{Specifications}
Freshkeeper provides users with a variety of features to make early morning delivery convenient and fresh while using their LG refrigerators. The Freshkeeper is a mini fridge that can be placed near the front door for early morning delivery users. As the number of consumers using early morning delivery has exploded, the amount of waste generated has become a social problem. Therefore, we want to present a mini portable refrigerator that provides refrigeration by using our service, and appeal to the fact that it can be stored anywhere at the moment when refrigeration is needed. 
When users stored in the database access the application, they are provided with the following key features. \\
(The app screen hasn't been configured yet, so we couldn't attach it. Therefore, Our team put the overall structure in the form of a graphical diagram. And we will explain the main functions and update it with images as soon as the screen configuration is completed. )
\begin{figure}[htbp]
\centerline{\includegraphics[width=9cm,height=9cm]{다이어그램.jpg}}
\caption{Overall Process}
\label{fig}
\end{figure}

\subsection{login and sign-up}
This page requires users to sign up to create an account and then sign in to use this service. The login page is used for user authentication. Because the application handles the user's shipping order information, estimated time of arrival, etc.
user's information, all user's information associated with the shipping service must be stored based on a unique email user account.

\subsection{Main Page}
The main page will provide a dashboard of frequently requested items at a glance, and we'll also be adding a quick launch feature for the NUGU AI speaker.

\subsection{Healthy Dishes Generator}
Utilizing ChatGPT, a generative AI, we plan to add an option to load the user's order history and select healthy ingredients that can be substituted with healthy ingredients to create a "Healthy Dishes Generator" feature. \par
 ex ) If you ordered sugar → How about using stevia instead next time? Or Butter → vegetable oil 
This is a feature that allows user to 'input' purchases list and either 1) create a healthy meal recipe or 2) learn from user's past orders and create healthier alternatives. 
\subsection{Mini Fridge (Freshkeeper) to Main Fridge (LG model) Integration}
We can utilize mobile APIs and sensors to check what's inside the mini fridge.Real time check: Check the items inside using the weight sensor.
Notification: If the weight sensor is 'true' for a certain period of time send a notification (like checking the mailbox)
Sleep mode: Customize the mini fridge to run when no deliveries are scheduled, etc.
Wake-up notification: When the user wakes up in the morning and opens the main fridge, know that the user is awake and notify the mini fridge that an item has been delivered (send a push message) or through the speaker.

\subsection{Coupang OPEN API Integration}
To connect to Coupang OPEN API, we will develop according to the guide, such as searching for products and checking product registration status on the Coupang Developers site. This feature allows that user can see the status of an item, whether it's being prepared, refunded, or placed on order. We're also working on a technology that, when applied to 'sleep mode', will check the estimated arrival time and turn on the refrigerator in time to avoid disturbing you in the wee hours of the morning. If the product is delivered but the weight sensor is false, we are also thinking about checking the image to prevent non-delivery errors. 
\subsection{Utilizing NUGU AI Speaker}
Utilizing the NUGU SDK provided by SK Telecom's NUGU developers, we will create AI services by connecting the NUGU platform to internet-connected devices and our applications. 
In our FreshKeeper mobile app, users can check the estimated arrival time based on voice recognition, receive an alert if the door is not closed, and remotely open the door of the mini fridge. 
\subsection{My Page / Settings}
On this page, user can set privacy settings or link LG Refrigerator device here.
\end{document}
