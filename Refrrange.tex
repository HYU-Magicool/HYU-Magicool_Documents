\documentclass[conference]{IEEEtran}
\IEEEoverridecommandlockouts
% The preceding line is only needed to identify funding in the first footnote. If that is unneeded, please comment it out.
\usepackage{cite}
\usepackage{amsmath,amssymb,amsfonts}
\usepackage{algorithmic}
\usepackage{graphicx}
\usepackage{textcomp}
\usepackage{xcolor}
\def\BibTeX{{\rm B\kern-.05em{\sc i\kern-.025em b}\kern-.08em
    T\kern-.1667em\lower.7ex\hbox{E}\kern-.125emX}}
\begin{document}

\title{Refrrange\\
{\footnotesize \large Refrigerator Arrangement Advisor Application with LG Refrigerators}}

\author{\IEEEauthorblockN{1\textsuperscript{st} Lee JunYoung}
\IEEEauthorblockA{\textit{dept. of Information Systems} \\
\textit{Hanyang University}\\
Seoul, Republic of Korea \\
leewnsdud123@gmail.com}
\and
\IEEEauthorblockN{2\textsuperscript{nd} Park SeokWon}
\IEEEauthorblockA{\textit{dept. of Information Systems} \\
\textit{Hanyang University}\\
Seoul, Republic of Korea \\
stevenpark00@hanyang.ac.kr}
}
\maketitle

\begin{abstract}
Within LG smarthome service, our team try to develop a refrigerator arrangement advice application which provide user-optimized recommendations. Refrrange, the service what we are to implement, collect the structure of customer refrigerator model and calculate the optimized composition of substances consist the inside. Plus, give notifications of expiration date about each items stored to keep customers consume foods more effectively and aware the importance with being healthy by safe food materials. The target of Refrrange can be explained as two words, 'optimization' and 'convenience.' Much of users could take advantage of Refrrange to make their refrigerator space efficiently by creating much empty spaces able to contain more items. Also, inconvenient user interface and too abstract advice image would turn refrigerator more chaotic, so prevention uncomfortable instances should be avoided. As a result, Refrrange will provide space-effective with convenient application usage to refrigerator owners.
\end{abstract}

\begin{IEEEkeywords}
refrigerator, space optimization, mobile application
\end{IEEEkeywords}

\begin{table}[h]
\caption{Role Assignment}
\begin{tabular}{| p{1.7cm}|p{1.5cm}|p{4.3cm} |}

\hline
Roles & Name & Role Description\\
\hline
Development \par Manager, \par User 
& Lee \par JunYeong 
& Development Manager is responsible for creating a clear project plan. This involves defining project goals, objectives, scope, and timeline. He ensures that the project development process is being done well, and makes sure that testing and implementation of the product is well deployed. \par
Users play an indirect role in shaping the app's requirements. Their needs and preferences provide the foundation for what features and functionalities the app should have. Developers and designers rely on user feedback and market research to determine the app's scope. Users are often the primary source of feedback during the app's development and after it's released. Developers use this feedback to refine and improve the app. Also, When users encounter issues or bugs in the app, they can report these problems to the development team. This feedback is essential for resolving issues and ensuring the app's stability.\\
\hline
\end{tabular}
\end{table}
\begin{table}[h]
\begin{tabular}{| p{1.7cm}|p{1.5cm}|p{4.3cm} |}
\hline
Software  \par Developer,  \par Customer 
& Park \par SeokWon 
& Software	 Developers are responsible for designing the overall architecture of the mobile app. They decide on the technology stack, frameworks, and the overall structure of the app.
Writing code is a primary responsibility. Developers use programming languages to create the app's functionality. They need to ensure that the code is efficient, maintainable, and follows best practices. Also, Software Developers implement the user interface (UI) and user experience (UX) components of the app. This involves creating responsive and visually appealing layouts. In addition, Software Developers are responsible for testing the app thoroughly to identify and fix bugs and issues.
\par Customer identify what they want to implement with the app when it comes to refrigerator management, and further identify and present what functions should be added to make refrigerator management more convenient and efficient. Such preferences of customers’ become the main objective for mobile app service. \\
\hline

\end{tabular}
\end{table}

\section{Introduction}
Since the first one was developed, refrigerator has been an essential device that cannot be separated from human life. Nowadays, refrigerators became so important that foods without refrigerator stored is totally rarely, but a critical weakness exists: narrow space.
We could be inexperienced to arrange contents inside the refrigerator, also classify some foods that influence others, even we may forget the existence of food or miss the position.
There are certain mobile applications helping people managing their refrigerators on the market, but their functions seem to be limited to only managing expiration date of the foods inside the refrigerator. 
‘Refrrange’ is an app that advise how to rearrange the substances with LG refrigerators. It might be helpful for users who admire to keep foods safe and fresh or feel hard to manage refrigerator space. Refrrange will identify the model of refrigerator and recognize items occupying refrigerator space to offer customers recommendations how to sort items effectively or visually by the user side.



\subsection{Significance of Refrigerators in Modern Life}
    
To appreciate the value of Refrrange, it's important to recognize the pivotal role refrigerators play in modern life. Refrigerators have become a staple in households around the world. They have transformed the way we store and consume food, ensuring that fresh and perishable items remain safe for consumption. Refrigerators have become an integral part of our daily lives, allowing us to keep a wide range of foods and beverages readily accessible. As a result, it's rare to find a household without a refrigerator.

However, this technological marvel comes with its own set of challenges. The most prominent among these challenges is the issue of limited space. Despite the generous storage capacity of modern refrigerators, users often find it difficult to efficiently organize their food items. This can lead to several problems:

\begin{itemize}
    \item Inefficient Arrangement\par
    Users might be inexperienced in arranging the contents of their refrigerators. They may struggle to categorize different foods or may inadvertently store items in a way that influences others negatively. The consequence is a cluttered and poorly organized fridge.    
    \item Food Wastage\par
    Inefficient organization often results in users forgetting about the existence of certain food items. These items may get pushed to the back of the refrigerator, where they are eventually forgotten and go to waste. This not only leads to financial losses but also contributes to food wastage, a pressing global issue.    
    \item Difficulty in Finding Items\par
    A chaotic refrigerator can be frustrating to deal with. When you're unable to locate specific items, it not only wastes time but can also lead to unnecessary stress and annoyance.
\end{itemize}

\subsection{Motivation}

The advent of smart home technology has brought about a paradigm shift in the way we interact with our everyday appliances. Refrigerators, once considered a humble and indispensable part of our lives, have now been reimagined as sophisticated devices in the era of LG's smarthome services. As part of this transformative journey, the development of the Refrrange application emerges as a remarkable innovation, one that promises to enhance the way we manage the contents of our refrigerators.

Refrrange is envisioned as an ingenious solution that delves into the intricate world of refrigerator management. It goes beyond the conventional purpose of a refrigerator as a mere storage space and adds a layer of intelligence to it. This application analyzes the structure of the customer's refrigerator model and calculates the optimized composition of substances within, while also providing valuable notifications about expiration dates. In doing so, it aims to revolutionize the way we consume food, fostering efficiency, health, and convenience.

\subsection{Objectives}
The objectives of Refrrange can be summarized in two simple yet profound words: 'optimization' and 'convenience.' Let's delve into how these objectives are achieved and the benefits they bring to users.
\begin{itemize}
    \item Optimization\par
    Refrrange seeks to optimize the way we use our refrigerator space. Modern refrigerators are a marvel of engineering, offering ample storage capacity, yet their potential often goes underutilized. Many users struggle with inefficiently organized contents, which not only waste space but also make it difficult to locate items when needed. Refrrange comes to the rescue by providing recommendations on how to rearrange the contents of the refrigerator to create more empty spaces, allowing users to store more items effectively. This optimization ensures that users get the most out of their refrigerator's capacity and minimizes food wastage.
    \item Convenience\par
    A cluttered refrigerator can be frustrating and chaotic to deal with. An inconvenient user interface and vague advice on food placement can exacerbate the problem. Refrrange takes convenience seriously by offering a user-friendly interface and clear, practical recommendations. Users are guided through the process of arranging their refrigerator, making it a hassle-free and enjoyable experience. The application aims to prevent uncomfortable instances by providing precise, easy-to-follow suggestions. The result is a refrigerator that is not only space-efficient but also a pleasure to use.
\end{itemize}

\subsection{Problem Statement}
\begin{itemize}
    \item Food Labeling and Categorization\par
    Our application should accurately categorize and label various food items to provide relevant recommendations. 
    \item Data Privacy and Security\par
    Because of the data analysis, user data collection is essential on working app normally. Security and privacy of user data should be ensured when the app deals with information such as user food inventory or consumption preference.
    \item Accuracy of Expiration Dates\par
    Expiration Dates are directly related to user health. Not to offer wrong expiration date is the most important problem Refrrange deals with.
    \item Notification and Reminder\par
    Timely delivery notifications and reminders is considerable in user-friendly aspect. Inform upcoming expiration dates have to be available for users.
    \item Inventory Tracking and Update\par
    Similar to expiration dates problem, tracking of refrigerator storage is in consideration how much value is on either the importance and the accuracy of inventory.
    \item User Interface and User Experience\par
    User-friendly interface and clear visualization of advice is another problem we got.
\end{itemize}

\subsection{Research on Related Software}
Samsung Family Hub\par
Samsung Family Hub is a part of Samsung's smart home ecosystem and focuses on enhancing user refrigerator experience. Family Hub offers a refrigerator management application which allows user to monitor and manage the contents of refrigerator in conjunction with the Samsung refrigerator.
\begin{itemize}
    \item Integration with Smart Refrigerators\par
    Connect with the Samsung Family Hub refrigerator to check and manage the refrigerator's contents in real-time.
    \item Ingredient Management\par
    Track the ingredients inside your refrigerator and receive notifications about expiration dates.
    \item Recipe Recommendations\par
    Based on the refrigerator's contents, it offers recipe recommendations to provide cooking ideas.
\end{itemize}
Whirlpool KitchenAid\par
Whirlpool offers a refrigerator management application that works with Whirlpool and KitchenAid appliances. The application focus on enhancing food preservation and offering convenient features. 
\begin{itemize}
    \item Refrigerator Status Monitoring\par
    Monitor the temperature and humidity of the refrigerator to maintain freshness.
    \item Expiration Date Tracking\par
    Track the expiration dates of food items in the refrigerator and provide alerts.
    \item Recipe Management\par
    Search and manage recipes based on the refrigerator's contents.
\end{itemize}
ThinQ\par
LG ThinQ is a broader application that integrates various LG smart devices, including refrigerators.
\begin{itemize}
    \item Smart Home Management\par
    LG ThinQ allows you to manage LG smart devices and refrigerators in one place.
    \item Food Management\par
    Monitor the contents of your refrigerator and receive notifications about expiration dates.
    \item Energy Savings\par
    Optimize and save energy consumption through refrigerator settings.
\end{itemize}


\section{Requirements}

\subsection{Mobile Application}
The primary requirement is the development of a mobile application specifically designed for the Android OS. The purpose of this application is to provide users with convenient access to the refrigerator management service.

\subsection{Refrigerator Registration}
The application should have the capability to retrieve information related to LG refrigerators from a central server. Alternatively, users should be able to manually register their refrigerators within the application.

\subsection{Adding Items}
The core functionality involves the addition of items to be stored within the user's refrigerator. The application should present a predefined list of commonly stored items in refrigerators, including essential details such as item name and expiration date. Additionally, users should be able to manually input items and their associated information if they are not found in the predefined list.

\subsection{Item information modification}
Users must have the option to modify the information of any specific item that they have added to their refrigerator. This editing feature ensures that the data remains accurate and up-to-date.

\subsection{Item Arrangement Helper}
The application offers a valuable feature that provides recommendations on how to arrange items within the refrigerator. These recommendations are tailored to the available space in the refrigerator and can take two forms:
\begin{itemize}
    \item Optimal Arrangement\par
    Suggests the most efficient way to store items for space optimization.
    \item Full Arrangement\par
    Recommends arrangements that maximize the capacity of the refrigerator.
\end{itemize}

\subsection{Notification}
The application includes a notification system that alerts users to the upcoming expiration dates of items stored in their refrigerators. Users can customize their notification preferences, including setting notification intervals, such as receiving alerts one week prior to an item's expiration date.

\subsection{Providing Information}
The application serves as an information resource, offering valuable insights and tips to assist users in effectively managing the contents of their refrigerators. This information can cover various aspects of refrigerator management, from food safety guidelines to best practices for organization and storage.

%\section{Development Environment}

%\subsection{Choice of software development platform}
%\subsection{Software in use}
%\subsection{Task distribution}

%\section{Specifications}

%\subsection{Mobile Application}
%\subsection{Refrigerator Registration}
%\subsection{Adding Items}
%\subsection{Item information modification}
%\subsection{Item Arrangement Helper}
%\subsection{Notification}
%\subsection{Providing Information}

\end{document}